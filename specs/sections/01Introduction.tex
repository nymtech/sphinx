\section{Introduction}\label{sec:introduction}

Sphinx is a cryptographic packet format introduced by Danezis and Goldberg [Add Ref] used to rely packets via multi-hop paths in decentralized networks.

\subsection{Terminology}\label{sec:terminology}
\begin{itemize}
    \item Message -  A variable-length sequence of bytes transmitted through a decentralized network.
    \item Packet - A fixed-length sequence of bytes transmitted through a decentralized network, containing the encrypted message and routing instructions.
    \item Header - A fixed-length part of a packet consisting of several components, which convey the information necessary to verify packet integrity and correctly process the packet.
    \item Payload - A fixed-length part of a packet containing an encrypted message.
    \item Group - A finite set of elements and a binary operation that satisfy the properties of closure, associativity, invertability, and the presence of an identity element.
    \item Group element - An individual element of a group.
    \item Group Order - The number of elements present in the group. 
    \item Group generator - A group element capable of generating any other element of the group, via repeated applications of the generator and the group operation. 
    \item Cyclic Group - A group generated by a single element.
    \item Tag - A cryptographic checksum on data that detects accidental or intentional modifications of the data.
\end{itemize}

\subsection{Conventions Used in This Document}
The key words "MUST", "MUST NOT", "REQUIRED", "SHALL", "SHALL NOT", "SHOULD", "SHOULD NOT", "RECOMMENDED",  "MAY", and "OPTIONAL" in this document are to be interpreted as described in RFC 2119 [Add Ref].

\subsection{Notation}\label{sec:notation}
\begin{itemize}
    \item x || y denotes the concatenation of x and y,
    \item x \^{} y denotes the bitwise XOR of x and y,
    \item v[a:b] denotes the sub-vector of v where a/b denote the start/end byte indexes (inclusive-exclusive),
    \item v[a:] denotes a sub-vector of v from index a (inclusive) till the last index (inclusive)
    \item v[:a] denotes a sub-vector of v from the first index (inclusive) till index a (exclusive)
    \item v[i] denotes the i'th element of list v,
    \item v.len denotes the length of vector v,
\end{itemize}

