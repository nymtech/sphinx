\section{Sphinx Packet Structure}\label{sec:structure}

A Sphinx Packet is composed of two segments: a header, containing routing instructions for the intermediate nodes, and a payload hiding the message content. The byte length of the Sphinx header is specified by parameter \textbf{HEADER\_LENGTH}, while the byte length of the Sphinx payload is defined by the parameter \textbf{PAYLOAD\_LENGTH}. The total size of a Sphinx packet is defined by \textbf{PACKET\_LENGTH}.
\begin{minted}{rust}
    struct SphinxPacket {
        header : [u8; HEADER_SIZE],
        payload: [u8; PAYLOAD_SIZE],
    }
\end{minted}

The header of a Sphinx packet contains routing instructions for the intermediate nodes and information necessary to verify packet integrity. The payload contains encrypted message content.

\subsection{Sphinx Header Overview}\label{sec:header}

The header of a Sphinx packet comprises of the following parts: an element of a cyclic group of prime order, encrypted routing information and an integrity authentication tag covering the encrypted routing information. A plaintext vector with additional information e.g., software version is attached at the beginning of the header.

\begin{minted}{rust}
    struct SphinxHeader {
        additional_data: [u8; AD_LENGTH],
        group_element: [u8; GROUP_ELEMENT_SIZE],
        encrypted_routing_information [u8; ENCRYPTED_ROUTING_INFO_SIZE],
        integrity_tag: [u8; HEADER_INTEGRITY_MAC_SIZE],
    }
\end{minted}

The group element is used by each mix in the packet route to derive a \textbf{secret} that is shared with the original sender of the packet. A secure key derivation function \kdf with input \textbf{secret} is used to further extract an encryption key \hek, integrity key \ik, \blind and payload key \pk. Value \blind is used by a relay node to blind the \alp to prevent packet linking while it traverses the route. The integrity key \ik is used to derive a hash-based message authentication code using \bet which is next compared against \intag to ensure no part of the header containing routing information has been modified.
The key \hek is used to remove layer of symmetric encryption from \bet to extract the new routing instruction \bet and new \intag for the next relay. The payload key \pk is used to remove a layer of encryption from the payload part.




